\documentclass[a4paper]{article}

\input{style/ch_xelatex.tex}
\input{style/scala.tex}

%代码段设置
\lstset{numbers=left,
basicstyle=\tiny,
numberstyle=\tiny,
keywordstyle=\color{blue!70},
commentstyle=\color{red!50!green!50!blue!50},
frame=single, rulesepcolor=\color{red!20!green!20!blue!20},
escapeinside=``
}

\graphicspath{ {images/} }
\usepackage{ctex}
\usepackage{graphicx}
\usepackage{color,framed}%文本框
\usepackage{listings}
\usepackage{caption}
\usepackage{amssymb}
\usepackage{enumerate}
\usepackage{xcolor}
\usepackage{bm}
\usepackage{lastpage}%获得总页数
\usepackage{fancyhdr}
\usepackage{tabularx}
\usepackage{geometry}
\usepackage{minted}
\usepackage{graphics}
\usepackage{subfigure}
\usepackage{float}
\usepackage{pdfpages}
\usepackage{pgfplots}
\pgfplotsset{width=10cm,compat=1.9}
\usepackage{multirow}
\usepackage{footnote}
\usepackage{booktabs}

%-----------------------伪代码------------------
\usepackage{algorithm}
\usepackage{algorithmicx}
\usepackage{algpseudocode}
\floatname{algorithm}{Algorithm}
\renewcommand{\algorithmicrequire}{\textbf{Input:}}
\renewcommand{\algorithmicensure}{\textbf{Output:}}
\usepackage{lipsum}
\makeatletter
\newenvironment{breakablealgorithm}
  {% \begin{breakablealgorithm}
  \begin{center}
     \refstepcounter{algorithm}% New algorithm
     \hrule height.8pt depth0pt \kern2pt% \@fs@pre for \@fs@ruled
     \renewcommand{\caption}[2][\relax]{% Make a new \caption
      {\raggedright\textbf{\ALG@name~\thealgorithm} ##2\par}%
      \ifx\relax##1\relax % #1 is \relax
         \addcontentsline{loa}{algorithm}{\protect\numberline{\thealgorithm}##2}%
      \else % #1 is not \relax
         \addcontentsline{loa}{algorithm}{\protect\numberline{\thealgorithm}##1}%
      \fi
      \kern2pt\hrule\kern2pt
     }
  }{% \end{breakablealgorithm}
     \kern2pt\hrule\relax% \@fs@post for \@fs@ruled
  \end{center}
  }
\makeatother
%------------------------代码-------------------
\usepackage{xcolor}
\usepackage{listings}
\lstset{
breaklines,%自动换行
basicstyle=\small,
escapeinside=``,
keywordstyle=\color{ blue!70} \bfseries,
commentstyle=\color{red!50!green!50!blue!50},%
stringstyle=\ttfamily,%
extendedchars=false,%
linewidth=\textwidth,%
numbers=left,%
numberstyle=\tiny \color{blue!50},%
frame=trbl%
rulesepcolor= \color{ red!20!green!20!blue!20}
}

%-------------------------页面边距--------------
\geometry{a4paper,left=2.3cm,right=2.3cm,top=2.7cm,bottom=2.7cm}
%-------------------------页眉页脚--------------
\usepackage{fancyhdr}
\pagestyle{fancy}
\lhead{\kaishu \leftmark}
% \chead{}
\rhead{\kaishu 机器学习实验报告}%加粗\bfseries
\lfoot{}
\cfoot{\thepage}
\rfoot{}
\renewcommand{\headrulewidth}{0.1pt}
\renewcommand{\footrulewidth}{0pt}%去掉横线
\newcommand{\HRule}{\rule{\linewidth}{0.5mm}}%标题横线
\newcommand{\HRulegrossa}{\rule{\linewidth}{1.2mm}}
\setlength{\textfloatsep}{10mm}%设置图片的前后间距
%--------------------文档内容--------------------

\begin{document}
\renewcommand{\contentsname}{目\ 录}
\renewcommand{\appendixname}{附录}
\renewcommand{\appendixpagename}{附录}
\renewcommand{\refname}{参考文献}
\renewcommand{\figurename}{图}
\renewcommand{\tablename}{表}
\renewcommand{\today}{\number\year 年 \number\month 月 \number\day 日}

%-------------------------封面----------------
\begin{titlepage}
    \begin{center}
    \includegraphics[width=0.8\textwidth]{NKU.png}\\[1cm]
    \vspace{20mm}
		\textbf{\huge\textbf{\kaishu{计算机学院}}}\\[0.5cm]
		\textbf{\huge{\kaishu{机器学习实验报告}}}\\[2.3cm]
		\textbf{\Huge\textbf{\kaishu{决策分类器实验}}}

		\vspace{\fill}

    \centering
    \textsc{\LARGE \kaishu{姓名\ :\ 周重天}}\\[0.5cm]
    \textsc{\LARGE \kaishu{学号\ :\ 2311082}}\\[0.5cm]
    \textsc{\LARGE \kaishu{专业\ :\ 计算机科学与技术}}\\[0.5cm]

    \vfill
    {\Large \today}
    \end{center}
\end{titlepage}

\renewcommand {\thefigure}{\thesection{}.{\arabic{figure}}}%图片按章标号
\renewcommand{\figurename}{图}
\renewcommand{\contentsname}{目录}
\cfoot{\thepage\ of \pageref{LastPage}}%当前页 of 总页数

% 生成目录
\clearpage
\tableofcontents
\newpage


\section{基础任务:ID3 决策树构建与预测(离散属性)}

\subsection{ID3 算法简介}

ID3(Iterative Dichotomiser 3)决策树是由 Ross Quinlan 提出的经典算法,仅适用于离散属性。ID3 使用信息增益作为属性选择准则,通过递归地选择增益最大的属性进行分裂,直至满足停止条件。

ID3 决策树的核心思想是:选择能够最大程度减少数据集不纯性(熵)的属性作为分裂属性。

\subsection{关键算法步骤}

\subsubsection{第一步:信息熵计算}

对于数据集 $S$,信息熵的计算公式为:

\begin{equation}
H(S) = -\sum_{i=1}^{c} p_i \log_2(p_i)
\end{equation}

其中 $p_i$ 表示类别 $i$ 在数据集中的比例,$c$ 为类别总数。

本实验中的实现如下:

\begin{lstlisting}[language=Python,caption={信息熵计算}]
def entropy(self, labels):
    """计算信息熵 H(S) = -∑ p_i * log2(p_i)"""
    if len(labels) == 0:
        return 0.0
    value_counts = Counter(labels)
    entropy_val = 0.0
    total = len(labels)
    for count in value_counts.values():
        if count > 0:
            p = count / total
            entropy_val -= p * np.log2(p)
    return entropy_val
\end{lstlisting}

\subsubsection{第二步:信息增益计算}

对于属性 $A$,其信息增益定义为:

\begin{equation}
\text{Gain}(S, A) = H(S) - \sum_{v \in \text{Values}(A)} \frac{|S_v|}{|S|} H(S_v)
\end{equation}

其中 $S_v$ 表示属性 $A$ 取值为 $v$ 的样本子集。

实现代码:

\begin{lstlisting}[language=Python,caption={信息增益计算}]
def information_gain(self, parent_labels, child_labels_list):
    """
    计算信息增益
    Gain(S, A) = Entropy(S) - ∑ |S_v|/|S| * Entropy(S_v)
    """
    parent_entropy = self.entropy(parent_labels)
    total_samples = len(parent_labels)
    if total_samples == 0:
        return 0.0
    weighted_child_entropy = 0.0
    for child_labels in child_labels_list:
        if len(child_labels) > 0:
            weight = len(child_labels) / total_samples
            weighted_child_entropy += weight * self.entropy(child_labels)
    return parent_entropy - weighted_child_entropy
\end{lstlisting}

\subsubsection{第三步:递归建树}

算法遍历所有离散属性,计算各属性的信息增益,选择增益最大的属性作为当前节点的分裂属性。然后按该属性的不同取值划分样本子集,对每个子集递归调用建树算法。

递归停止条件:
\begin{enumerate}
\item 子集中所有样本属于同一类别
\item 没有可用属性可进行分裂
\item 子集为空
\end{enumerate}

核心递归代码片段:

\begin{lstlisting}[language=Python,caption={ID3 递归建树}]
def build_tree(self, X, y, available_features):
    node = TreeNode()
    node.samples = len(y)
    node.class_distribution = dict(Counter(y))
    node.majority_class = Counter(y).most_common(1)[0][0] if len(y) > 0 else None
    
    # 停止条件 1:标签全相同
    if len(np.unique(y)) <= 1:
        node.is_leaf = True
        node.prediction = y.iloc[0] if len(y) > 0 else None
        return node
    
    # 停止条件 2 和 3:没有属性或子集为空
    if len(available_features) == 0 or len(X) == 0:
        node.is_leaf = True
        node.prediction = node.majority_class
        return node
    
    # 选择最优分裂属性(信息增益最大)
    best_feature = None
    best_gain = -1
    for feature in available_features:
        feature_values = X[feature].unique()
        child_labels_list = []
        for value in feature_values:
            mask = X[feature] == value
            child_labels = y[mask]
            child_labels_list.append(child_labels.values)
        gain = self.information_gain(y.values, child_labels_list)
        if gain > best_gain:
            best_gain = gain
            best_feature = feature
    
    # 递归构建子树
    remaining_features = [f for f in available_features if f != best_feature]
    node.feature = best_feature
    node.children = {}
    for value in X[best_feature].unique():
        mask = X[best_feature] == value
        X_subset = X[mask].reset_index(drop=True)
        y_subset = y[mask].reset_index(drop=True)
        node.children[value] = self.build_tree(X_subset, y_subset, remaining_features)
    return node
\end{lstlisting}

\subsubsection{代码讲解}

递归建树函数 \texttt{build\_tree} 是 ID3 算法的核心,其执行流程如下:

\begin{enumerate}
\item \textbf{节点初始化}:创建新的树节点,计算该节点所有样本的多数类。这用于处理无法完全分裂的情况下的默认预测。

\item \textbf{停止条件检查}:
\begin{itemize}
    \item 若所有样本标签相同(\texttt{len(np.unique(y)) <= 1}),则该节点是纯净的,无需继续分裂,直接创建叶子节点。
    \item 若没有可用属性(\texttt{len(available\_features) == 0})或样本子集为空,也无法继续分裂,创建叶子节点并使用多数类作为预测。
\end{itemize}

\item \textbf{属性选择}:遍历所有可用属性,对每个属性计算其信息增益。具体步骤为:
\begin{itemize}
    \item 获取该属性的所有不同取值(\texttt{X[feature].unique()})
    \item 按各取值将样本分组,构建子集标签列表
    \item 调用 \texttt{information\_gain} 方法计算该属性的增益
    \item 记录增益最大的属性作为最优分裂属性
\end{itemize}

\item \textbf{递归分裂}:选定最优分裂属性后,按该属性的每个不同取值创建子分支。对于每个分支:
\begin{itemize}
    \item 使用布尔掩码 \texttt{mask} 筛选出满足该属性值的样本
    \item 从可用属性集合中移除已分裂的属性(\texttt{remaining\_features})
    \item 递归调用 \texttt{build\_tree} 构建子树
\end{itemize}

\end{enumerate}

\subsection{实验结果}

使用 Watermelon-train1.csv(纯离散属性)训练 ID3 决策树,对 Watermelon-test1.csv 进行预测。

\textbf{测试集结果:}

\begin{table}[H]
\centering
\caption{ID3 决策树在 test1 上的性能}
\begin{tabular}{|c|c|}
\hline
\textbf{指标} & \textbf{值} \\
\hline
训练集样本数 & 16 \\
\hline
测试集样本数 & 10 \\
\hline
特征维数 & 5 \\
\hline
分类精度 & 0.70 (70\%) \\
\hline
\end{tabular}
\end{table}

\begin{figure}[H]
\centering
\includegraphics[width=0.8\textwidth]{fig/ID3.png}
\caption{ID3 决策树实验运行结果}
\label{fig:id3_result}
\end{figure}

\subsection{关键代码:预测函数}

\begin{lstlisting}[language=Python,caption={ID3 预测}]
def predict_one(self, x):
    """预测单个样本"""
    node = self.tree
    while not node.is_leaf:
        feature = node.feature
        feature_value = x.get(feature, None)
        # 如果特征值不在子节点中,使用多数类回退
        if feature_value not in node.children:
            return node.majority_class
        node = node.children[feature_value]
    return node.prediction
\end{lstlisting}

\section{中级要求:C4.5 决策树构建与预测}

\subsection{C4.5 算法改进简介}

C4.5 是对 ID3 的重要改进,主要区别在于:

\begin{enumerate}
\item 支持连续属性的处理
\item 使用信息增益率代替信息增益,避免对多值属性的偏向
\end{enumerate}

\subsection{连续属性处理方法}

对于连续属性,C4.5 采用如下策略:

将属性值从小到大排序,相邻两个不同值的中点作为候选分裂阈值。对每个阈值,将数据分为两部分:$A \leq t$ 和 $A > t$。

\begin{lstlisting}[language=Python,caption={连续属性最优阈值寻找}]
def find_best_continuous_split(self, X_col, y, feature_name):
    """
    为连续属性寻找最优分裂阈值
    候选阈值:排序后相邻不同值的中点
    """
    X_col_sorted = sorted(set(X_col.values))
    if len(X_col_sorted) <= 1:
        return None, -1
    thresholds = []
    for i in range(len(X_col_sorted) - 1):
        threshold = (X_col_sorted[i] + X_col_sorted[i + 1]) / 2
        thresholds.append(threshold)
    best_threshold = None
    best_gain_ratio = -1
    for threshold in thresholds:
        left_mask = X_col <= threshold
        right_mask = X_col > threshold
        y_left = y[left_mask]
        y_right = y[right_mask]
        if len(y_left) == 0 or len(y_right) == 0:
            continue
        child_labels_list = [y_left.values, y_right.values]
        gain_ratio = self.information_gain_ratio(y.values, child_labels_list)
        if gain_ratio > best_gain_ratio:
            best_gain_ratio = gain_ratio
            best_threshold = threshold
    return best_threshold, best_gain_ratio
\end{lstlisting}

\subsubsection{代码讲解}

连续属性处理是 C4.5 相比 ID3 的关键改进。\texttt{find\_best\_continuous\_split} 函数的核心思想是:

\begin{enumerate}
\item \textbf{候选阈值生成}:对属性值进行排序并提取唯一值。相邻两个不同值的中点被作为候选分裂阈值。例如,若排序后的唯一值为 $[0.5, 0.7, 0.9]$,则候选阈值为 $[0.6, 0.8]$。这种方式能有效减少阈值的搜索空间。

\item \textbf{阈值评估}:对每个候选阈值,将数据分为两部分:$X \leq \text{threshold}$ 和 $X > \text{threshold}$。计算此二分裂下的信息增益率。

\item \textbf{最优阈值选择}:在所有有效的二分裂中(两部分都非空),选择能最大化增益率的阈值。

\end{enumerate}

相比穷举所有可能的分裂点,该方法大大降低了计算复杂度,同时保留了充分的表达能力。

\subsection{信息增益率}

为避免 ID3 对多值属性的偏向,C4.5 引入信息增益率:

\begin{equation}
\text{GainRatio}(S, A) = \frac{\text{Gain}(S, A)}{\text{SplitInfo}(S, A)}
\end{equation}

其中分裂信息定义为:

\begin{equation}
\text{SplitInfo}(S, A) = -\sum_{v \in \text{Values}(A)} \frac{|S_v|}{|S|} \log_2 \frac{|S_v|}{|S|}
\end{equation}

\begin{lstlisting}[language=Python,caption={C4.5 信息增益率}]
def information_gain_ratio(self, parent_labels, child_labels_list):
    """
    计算信息增益率
    GainRatio(S, A) = Gain(S, A) / SplitInfo(S, A)
    """
    parent_entropy = self.entropy(parent_labels)
    total_samples = len(parent_labels)
    if total_samples == 0:
        return 0.0
    weighted_child_entropy = 0.0
    child_sizes = []
    for child_labels in child_labels_list:
        child_sizes.append(len(child_labels))
        if len(child_labels) > 0:
            weight = len(child_labels) / total_samples
            weighted_child_entropy += weight * self.entropy(child_labels)
    gain = parent_entropy - weighted_child_entropy
    split_info = self.split_info(child_sizes)
    if split_info == 0:
        return 0.0
    gain_ratio = gain / split_info
    return gain_ratio
\end{lstlisting}

\subsubsection{代码讲解}

信息增益率是对信息增益的改进,解决了 ID3 对多值属性的偏好问题。代码实现流程如下:

\begin{enumerate}
\item \textbf{基础增益计算}:首先计算属性分裂前的熵(\texttt{parent\_entropy}),然后计算分裂后各子集的加权熵,二者之差即为原始的信息增益。

\item \textbf{分裂信息计算}:\texttt{split\_info} 方法计算分裂信息,其值随着分支数增加而增加。多值属性会产生较大的分裂信息。

\item \textbf{增益率归一化}:通过将增益除以分裂信息,使得多值属性的增益被压低,避免算法过度偏好多值属性。这是 C4.5 相比 ID3 的根本改进。

\item \textbf{边界处理}:当分裂信息为 0(即所有分支大小都相等且只有一个分支)时,避免除以零的错误。

\end{enumerate}

在实践中,信息增益率能产生更平衡的决策树,对未见过的数据有更好的泛化能力。

\subsection{实验结果}

使用 Watermelon-train2.csv(离散+连续属性混合)训练 C4.5 决策树,对 Watermelon-test2.csv 进行预测。

\begin{table}[H]
\centering
\caption{C4.5 决策树在 test2 上的性能}
\begin{tabular}{|c|c|}
\hline
\textbf{指标} & \textbf{值} \\
\hline
训练集样本数 & 17 \\
\hline
测试集样本数 & 5 \\
\hline
特征维数 & 6 \\
\hline
其中连续属性数 & 2 \\
\hline
分类精度 & 0.60 (60\%) \\
\hline
\end{tabular}
\end{table}

\begin{figure}[H]
\centering
\includegraphics[width=0.8\textwidth]{fig/C4.5.png}
\caption{C4.5 决策树实验运行结果}
\label{fig:c45_result}
\end{figure}

\section{C4.5 算法属性处理过程}

\subsection{属性类型自动识别}

C4.5 利用 pandas 的 \texttt{is\_numeric\_dtype} 函数自动识别连续属性:

\begin{lstlisting}[language=Python,caption={属性类型自动识别}]
def fit(self, X, y):
    """训练 C4.5 决策树"""
    self.feature_names = X.columns.tolist()
    self.class_label = y.name if y.name else "label"
    # 自动识别连续属性
    from pandas.api.types import is_numeric_dtype
    for col in self.feature_names:
        if is_numeric_dtype(X[col]):
            self.continuous_features.add(col)
    available_features = [f for f in self.feature_names]
    self.tree = self.build_tree(X.reset_index(drop=True), 
                                y.reset_index(drop=True), 
                                available_features)
    return self
\end{lstlisting}

\subsubsection{代码讲解}

属性类型自动识别是 C4.5 实现的重要细节,使得算法能自动适应混合属性数据集:

\begin{enumerate}
\item \textbf{属性收集}:\texttt{fit} 方法首先提取数据集的所有特征列名,保存在 \texttt{feature\_names} 中。

\item \textbf{类型识别}:使用 pandas 库的 \texttt{is\_numeric\_dtype} 函数检查每个特征列的数据类型。若为数值型(int、float 等),则标记为连续属性,加入 \texttt{continuous\_features} 集合;否则视为离散属性。

\item \textbf{差异化处理}:在建树过程中,当遍历特征选择最优分裂属性时,会针对特征类型采用不同策略:
\begin{itemize}
    \item 连续特征:调用 \texttt{find\_best\_continuous\_split} 寻找最优二分阈值
    \item 离散特征:枚举所有不同取值,按值进行多路分裂
\end{itemize}

\item \textbf{用户友好性}:自动识别机制让用户无需手动预处理数据,直接传入混合属性数据集即可。相比 ID3 需要预先离散化连续属性,C4.5 更加易用。

\end{enumerate}

\section{高级要求:CART 决策树构建与预测}

\subsection{CART 算法简介}

CART(Classification And Regression Tree)由 Breiman 等人提出。CART 的主要特点:

\begin{enumerate}
\item 总是生成二叉树结构
\item 使用基尼指数(Gini Index)作为分裂准则
\item 支持离散和连续属性
\end{enumerate}

\subsection{基尼指数}

对数据集 $D$,基尼指数定义为:

\begin{equation}
\text{Gini}(D) = 1 - \sum_{k=1}^{c} p_k^2
\end{equation}

其中 $p_k$ 是类别 $k$ 在数据集中的比例。基尼指数反映数据集的不纯度。

\subsection{CART 建树算法}

CART 对连续属性和离散属性都采用二分策略。对于离散属性,每个节点分裂为"取值为 $v$"和"取值不为 $v$"两个分支。

\begin{lstlisting}[language=Python,caption={CART 基尼指数计算}]
def gini_index(self, labels):
    """计算基尼指数 Gini(D) = 1 - ∑ p_k^2"""
    if len(labels) == 0:
        return 0.0
    value_counts = Counter(labels)
    gini = 1.0
    total = len(labels)
    for count in value_counts.values():
        if count > 0:
            p = count / total
            gini -= p * p
    return gini

def weighted_gini(self, left_labels, right_labels):
    """计算二分后的加权基尼指数"""
    total = len(left_labels) + len(right_labels)
    if total == 0:
        return 0.0
    left_weight = len(left_labels) / total
    right_weight = len(right_labels) / total
    return (left_weight * self.gini_index(left_labels) + 
            right_weight * self.gini_index(right_labels))
\end{lstlisting}

\subsubsection{代码讲解}

CART 使用基尼指数作为分裂准则,相比信息论方法具有计算优势。代码涵盖两个关键函数:

\begin{enumerate}
\item \textbf{基尼指数计算}(\texttt{gini\_index}):
\begin{itemize}
    \item 遍历数据集中各类别,计算每类的比例 $p_k$
    \item 基尼指数定义为 $Gini(D) = 1 - \sum_{k=1}^{c} p_k^2$,体现数据的不纯度
    \item 当数据集只包含一个类别时,基尼指数为 0(纯净);当各类比例相等时,基尼指数最大
\end{itemize}

\item \textbf{加权基尼指数}(\texttt{weighted\_gini}):
\begin{itemize}
    \item 对于二分裂,分别计算左右两个子集的基尼指数
    \item 按子集大小比例进行加权求和
    \item 公式为:$Gini_{split} = \frac{|L|}{|D|} \cdot Gini(L) + \frac{|R|}{|D|} \cdot Gini(R)$
    \item 在寻找最优分裂时,选择能最小化加权基尼指数的分裂点
\end{itemize}

\end{enumerate}

CART 的二分策略(每次只分裂成两个分支)使得树的结构更规则,易于剪枝和理解。虽然可能导致树深度较深,但通过剪枝可以获得更优的复杂度-精度平衡。

\subsection{实验结果}

使用相同的 Watermelon-train2.csv 训练 CART 决策树。

\begin{table}[H]
\centering
\caption{CART 决策树在 test2 上的性能}
\begin{tabular}{|c|c|}
\hline
\textbf{指标} & \textbf{值} \\
\hline
训练集样本数 & 17 \\
\hline
测试集样本数 & 5 \\
\hline
特征维数 & 6 \\
\hline
树结构 & 二叉树 \\
\hline
分类精度 & 0.60 (60\%) \\
\hline
\end{tabular}
\end{table}

\begin{figure}[H]
\centering
\includegraphics[width=0.8\textwidth]{fig/CART.png}
\caption{CART 决策树实验运行结果}
\label{fig:cart_result}
\end{figure}

\section{高级要求:决策树剪枝}

\subsection{剪枝算法选择:简化后剪枝(REP)}

本实验采用简化后剪枝(Reduced Error Pruning, REP)算法。REP 是一种后剪枝方法,通过在验证集上评估剪枝效果来决定是否进行剪枝。

\subsection{REP 算法流程}

\begin{enumerate}
\item 从训练集中随机分出 20\% 作为验证集(random\_state=42 保证可复现性)
\item 在较小的训练集上重新训练决策树
\item 对每个内部节点,尝试将其替换为叶子(预测为该节点的多数类)
\item 若替换后验证集精度不下降,则保留该剪枝操作
\item 自底向上重复第 3、4 步,直到无法继续剪枝
\end{enumerate}

\begin{lstlisting}[language=Python,caption={简化后剪枝实现}]
def reduced_error_pruning(tree_model, X_valid, y_valid):
    """
    简化后剪枝 (Reduced Error Pruning, REP)
    """
    def calculate_accuracy(model, X, y):
        predictions = model.predict(X)
        return np.mean(predictions == y.values)
    
    def prune_node(node, X_valid, y_valid, model):
        """递归剪枝"""
        if node.is_leaf:
            return False
        # 先对所有子节点尝试剪枝
        any_pruned = False
        for child in node.children.values():
            if prune_node(child, X_valid, y_valid, model):
                any_pruned = True
        # 计算剪枝前的精度
        acc_before = calculate_accuracy(model, X_valid, y_valid)
        # 尝试将当前节点替换为叶子
        old_is_leaf = node.is_leaf
        old_prediction = node.prediction
        old_feature = node.feature
        old_children = node.children
        node.is_leaf = True
        node.prediction = node.majority_class
        node.feature = None
        node.children = {}
        # 计算剪枝后的精度
        acc_after = calculate_accuracy(model, X_valid, y_valid)
        # 如果精度没有下降,保留剪枝
        if acc_after >= acc_before:
            return True
        else:
            # 恢复原状
            node.is_leaf = old_is_leaf
            node.prediction = old_prediction
            node.feature = old_feature
            node.children = old_children
            return any_pruned
    
    for _ in range(100):  # 最多迭代 100 次
        if not prune_node(tree_model.tree, X_valid, y_valid, tree_model):
            break
    return tree_model
\end{lstlisting}

\subsubsection{代码讲解}

简化后剪枝(REP)通过在验证集上的精度变化来指导剪枝决策。实现原理如下:

\begin{enumerate}
\item \textbf{验证集分离}:在调用剪枝前,通常从训练集中分出 20\% 作为验证集(通过 \texttt{np.random.choice})。验证集用于评估剪枝效果,不参与树的构建。

\item \textbf{递归剪枝流程}(\texttt{prune\_node} 函数):
\begin{itemize}
    \item 若节点是叶子,无需剪枝,直接返回
    \item 否则,先对所有子节点递归尝试剪枝,自底向上进行
    \item 对当前节点,记录原始状态(\texttt{old\_*} 变量)
    \item 尝试将节点替换为叶子(预测为该节点的多数类),临时改变节点属性
    \item 在验证集上计算替换前后的精度(\texttt{acc\_before} 和 \texttt{acc\_after})
    \item 若替换后精度不下降,则保留剪枝;否则恢复原状
\end{itemize}

\item \textbf{迭代优化}:外层循环最多迭代 100 次,每次调用 \texttt{prune\_node}。只要还有节点被成功剪枝,就继续迭代。当某次迭代无任何剪枝发生,算法停止。

\item \textbf{关键特性}:
\begin{itemize}
    \item REP 是后剪枝方法,先构建完整的树再剪枝,避免过早停止导致信息丧失
    \item 使用独立的验证集而非训练集评估剪枝效果,更客观地反映泛化能力
    \item 自底向上的递归策略确保了剪枝的系统性和一致性
\end{itemize}

\end{enumerate}

\subsection{剪枝实施结果}

\subsubsection{ID3 树的剪枝}

\begin{table}[H]
\centering
\caption{ID3 决策树剪枝前后对比}
\begin{tabular}{|c|c|c|c|}
\hline
\textbf{指标} & \textbf{剪枝前} & \textbf{剪枝后} & \textbf{变化} \\
\hline
测试集精度 & 0.60 & 0.50 & $-0.10$ \\
\hline
节点数 & 14 & 1 & $-13$ \\
\hline
树深度 & 1 & 0 & $-1$ \\
\hline
\end{tabular}
\end{table}

\subsubsection{C4.5 树的剪枝}

\begin{table}[H]
\centering
\caption{C4.5 决策树剪枝前后对比}
\begin{tabular}{|c|c|c|c|}
\hline
\textbf{指标} & \textbf{剪枝前} & \textbf{剪枝后} & \textbf{变化} \\
\hline
测试集精度 & 0.60 & 0.60 & $0.00$ \\
\hline
节点数 & 3 & 3 & $0$ \\
\hline
树深度 & 1 & 1 & $0$ \\
\hline
\end{tabular}
\end{table}

\subsubsection{CART 树的剪枝}

\begin{table}[H]
\centering
\caption{CART 决策树剪枝前后对比}
\begin{tabular}{|c|c|c|c|}
\hline
\textbf{指标} & \textbf{剪枝前} & \textbf{剪枝后} & \textbf{变化} \\
\hline
测试集精度 & 0.60 & 0.60 & $0.00$ \\
\hline
节点数 & 3 & 3 & $0$ \\
\hline
树深度 & 1 & 1 & $0$ \\
\hline
\end{tabular}
\end{table}

\subsection{剪枝分析}

ID3 树的剪枝导致精度下降,说明该模型在验证集上的最优结构保留了所有节点。而 C4.5 和 CART 在剪枝前已经相对简洁,没有进一步的优化空间。这表明对于 Watermelon 数据集,原始的 C4.5 和 CART 模型已经具有良好的泛化能力。


\section{扩展要求:三种决策树算法比较分析}

\subsection{维度一:分裂准则}

\begin{table}[H]
\centering
\caption{三种算法的分裂准则对比}
\begin{tabular}{|c|c|c|c|}
\hline
\textbf{算法} & \textbf{分裂准则} & \textbf{公式} & \textbf{特点} \\
\hline
ID3 & 信息增益 & $\text{Gain}(S,A) = H(S) - \sum_v \frac{|S_v|}{|S|}H(S_v)$ & 偏向多值属性 \\
\hline
C4.5 & 信息增益率 & $\text{GainRatio}(S,A) = \frac{\text{Gain}(S,A)}{\text{SplitInfo}(S,A)}$ & 消除多值偏向 \\
\hline
CART & 基尼指数 & $\text{Gini}(D) = 1 - \sum_k p_k^2$ & 计算简便 \\
\hline
\end{tabular}
\end{table}

\subsection{维度二:树结构}

\begin{table}[H]
\centering
\caption{三种算法的树结构对比}
\begin{tabular}{|c|c|c|c|}
\hline
\textbf{算法} & \textbf{树结构类型} & \textbf{分裂策略} & \textbf{优缺点} \\
\hline
ID3 & 多叉树 & 多路分裂 & 结构紧凑,易理解;剪枝困难 \\
\hline
C4.5 & 多叉树 & 多路分裂 & 继承 ID3;支持连续属性 \\
\hline
CART & 二叉树 & 二分 & 易剪枝,规则明确;树可能较深 \\
\hline
\end{tabular}
\end{table}

\subsection{维度三:属性处理能力}

\begin{table}[H]
\centering
\caption{三种算法的属性处理对比}
\begin{tabular}{|c|c|c|c|}
\hline
\textbf{算法} & \textbf{离散属性} & \textbf{连续属性} & \textbf{自动识别} \\
\hline
ID3 & \checkmark 原生支持 & \texttimes 需预处理 & \texttimes \\
\hline
C4.5 & \checkmark 原生支持 & \checkmark 自动二分 & \checkmark \\
\hline
CART & \checkmark 二分处理 & \checkmark 自动二分 & \checkmark \\
\hline
\end{tabular}
\end{table}

\subsection{算法选择建议}

\begin{enumerate}
\item \textbf{ID3}:适用于纯离散属性数据集,数据规模较小,属性值有限的场景。优点是算法简洁高效,缺点是不支持连续属性。

\item \textbf{C4.5}:适用于混合属性数据集,属性数较多,类标签多的场景。通过增益率解决了 ID3 的多值偏向问题,扩展了连续属性支持。

\item \textbf{CART}:适用于生产环境,需要回归扩展的场景。二叉树结构易于剪枝和理解,基尼指数计算更快。

\end{enumerate}

\subsection{本实验观察}

在 Watermelon 数据集上,三种算法在 test2 上的精度都达到 0.60。C4.5 和 CART 都能有效处理连续属性(密度、含糖量等),而 ID3 仅限于离散属性。从模型复杂度看,C4.5 和 CART 生成的树都比较简洁(3 个节点),具有较好的泛化能力。

\section{附录:代码仓库}

本实验的全部源代码已上传至 GitHub 仓库:

\textbf{GitHub 仓库地址}: \url{https://github.com/sskystack/machinelearning.git}

源代码文件:\texttt{work6/exp\_6.py}



\end{document}
